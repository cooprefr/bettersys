\documentclass[11pt,a4paper]{article}

% Packages
\usepackage[utf8]{inputenc}
\usepackage[T1]{fontenc}
\usepackage{amsmath,amssymb,amsthm}
\usepackage{mathtools}
\usepackage{algorithm}
\usepackage{algpseudocode}
\usepackage{booktabs}
\usepackage{hyperref}
\usepackage{geometry}
\usepackage{enumitem}
\usepackage{graphicx}
\usepackage{xcolor}

\geometry{margin=1in}

% Theorem environments
\theoremstyle{definition}
\newtheorem{definition}{Definition}[section]
\newtheorem{assumption}{Assumption}[section]
\newtheorem{proposition}{Proposition}[section]
\newtheorem{remark}{Remark}[section]
\newtheorem{corollary}{Corollary}[section]

% Custom commands
\newcommand{\E}{\mathbb{E}}
\newcommand{\Var}{\mathrm{Var}}
\newcommand{\Cov}{\mathrm{Cov}}
\newcommand{\Q}{\mathbb{Q}}
\newcommand{\PP}{\mathbb{P}}
\newcommand{\R}{\mathbb{R}}
\newcommand{\N}{\mathbb{N}}
\DeclareMathOperator{\logit}{logit}
\DeclareMathOperator{\sigmoid}{sigmoid}

\title{\textbf{15-Minute Up/Down Crypto Markets on Polymarket} \\ \Large Measure-Theoretic and Microstructure-Aware Specification (FAST15M)}
\author{BetterBot Quantitative Team}
\date{January 2026}

\begin{document}

\maketitle

\begin{abstract}
This document specifies the implemented trading logic for Polymarket \emph{15-minute crypto Up/Down} markets. We begin from a filtered probability space and interpret conditional expectation as an $L^2$-orthogonal projection (functional-analytic view), then model the \emph{Up} event as a digital payoff and derive probability estimates under a local lognormal/GBM approximation. We also formalize the implemented RN-JD correction in log-odds space with belief-volatility tracking and jump-regime gating. Finally, we state explicit assumptions for Polymarket's \emph{15-minute fee schedule} (price-dependent quadratic fees), orderbook microstructure (bid/ask, staleness, IOC execution), and oracle basis risk (Chainlink vs Binance), and we present the complete decision algorithm.
\end{abstract}

\tableofcontents
\newpage

%==============================================================================
\section{Introduction and Problem Setting}
%==============================================================================

\subsection{Market Structure}

We consider binary prediction markets on Polymarket where the outcome is determined by whether a cryptocurrency's price ends higher or lower than its starting price over a 15-minute window.

\begin{definition}[15-Minute Up/Down Market]
Let $t_0$ denote the market start time and $t_1 = t_0 + 900$ seconds the expiry time. Let $P_t$ denote the price of the underlying asset (BTC, ETH, SOL, or XRP) at time $t$. The market resolves as:
\begin{equation}
    \text{Outcome} = \begin{cases}
        \texttt{Up} & \text{if } P^{\mathrm{CL}}_{t_1} > P^{\mathrm{CL}}_{t_0} \\
        \texttt{Down} & \text{if } P^{\mathrm{CL}}_{t_1} \leq P^{\mathrm{CL}}_{t_0}
    \end{cases}
\end{equation}
\end{definition}

\begin{remark}[Oracle vs observed price]
In implementation, the strategy observes a high-frequency Binance mid $P_t$.
Settlement uses a Chainlink oracle price $P_t^{\mathrm{CL}}$.
The polling FAST15M path may skip trading when estimated oracle lag/divergence is dangerous; otherwise, the model treats $P_t$ as a proxy for $P_t^{\mathrm{CL}}$.
\end{remark}

\begin{remark}
The ``equal'' case ($P_{t_1} = P_{t_0}$) resolves to \texttt{Down}. In practice, exact equality has measure zero under continuous price models.
\end{remark}

\subsection{Notation}

\begin{table}[h]
\centering
\begin{tabular}{cl}
\toprule
\textbf{Symbol} & \textbf{Description} \\
\midrule
$P_t$ & Underlying asset price at time $t$ \\
$P_t^{\mathrm{CL}}$ & Chainlink oracle settlement price at time $t$ \\
$P_0$ & Price at market start ($t = t_0$) \\
$P_{\text{now}}$ & Current observed price \\
$\tau$ & Time remaining to expiry (seconds) \\
$\sigma$ & Volatility per $\sqrt{\text{second}}$ \\
$p_{\text{up}}$ & Probability of \texttt{Up} outcome \\
$p_{\text{down}}$ & Probability of \texttt{Down} outcome ($= 1 - p_{\text{up}}$) \\
$a_{\text{up}}, a_{\text{down}}$ & Best ask prices for Up/Down contracts \\
$B$ & Bankroll (available capital in USDC) \\
$f^*$ & Kelly fraction \\
$\kappa$ & Fractional Kelly multiplier \\
$s$ & Shrinkage parameter \\
\bottomrule
\end{tabular}
\caption{Notation summary}
\end{table}

%==============================================================================
\section{Mathematical Setting: Prediction Markets as Traded Conditional Expectations}
%==============================================================================

\subsection{Filtered probability space (information model)}

\begin{definition}[Stochastic basis]\label{def:stochastic_basis}
Fix a finite horizon $T>0$. We work on a stochastic basis
$(\Omega,\mathcal{F},(\mathcal{F}_t)_{t\in[0,T]},\PP)$ satisfying the usual conditions.
The filtration $(\mathcal{F}_t)$ represents the \emph{information} available to the strategy at time $t$ (Binance mid-price feed, Polymarket orderbook state, oracle health signals, and internal clocks/latency measurements).
\end{definition}

\begin{remark}[Functional-analytic viewpoint]
For any probability measure $\mu$ on $(\Omega,\mathcal{F})$, the space $L^2(\Omega,\mathcal{F},\mu)$ is a Hilbert space.
For any $t$, the conditional expectation operator
$\E^{\mu}[\cdot\mid\mathcal{F}_t]$ is the $L^2$-orthogonal projection onto the closed subspace $L^2(\Omega,\mathcal{F}_t,\mu)$.
This is the precise sense in which ``pricing by conditional expectation'' is a projection of future payoffs onto currently observable information.
\end{remark}

\subsection{Contracts, payoff random variables, and the Up/Down event}

Let $t_0$ denote the window start, $t_1=t_0+900$ seconds the window end.
Let $P_t$ denote the \emph{observed/model} price used by the strategy (Binance mid) and let $P_t^{\mathrm{CL}}$ denote the \emph{settlement oracle} reference price (Chainlink).

\begin{definition}[Resolution random variables]\label{def:resolution_payoffs}
Define the event
\begin{equation}
  A \;:=\; \{ P^{\mathrm{CL}}_{t_1} > P^{\mathrm{CL}}_{t_0} \}\,.
\end{equation}
Define the Up payoff $Y^{\mathrm{Up}} := \mathbf{1}_A$ and the Down payoff $Y^{\mathrm{Down}} := 1-Y^{\mathrm{Up}}$.
\end{definition}

\begin{remark}
The code uses Binance for $P$ (fast, high frequency) but settlement uses Chainlink $P^{\mathrm{CL}}$.
This introduces \emph{basis/oracle risk} even if the model for $\widetilde{S}$ is perfect.
The polling FAST15M path implements an \emph{oracle lag/divergence} guardrail that can skip trading when the estimated basis is dangerous.
\end{remark}

\subsection{No-arbitrage, pricing measures, and what ``risk-neutral'' means here}

Polymarket outcome shares are bounded claims paying 1 USDC on the winning outcome and 0 otherwise.
Let $\pi_t^{\mathrm{Up}}$ and $\pi_t^{\mathrm{Down}}$ denote the (mid) prices at time $t$ for the Up/Down shares, respectively.

\begin{assumption}[Pricing measure (idealized)]\label{ass:pricing_measure}
Ignoring fees, spreads, and inventory constraints, there exists a probability measure $\Q\sim\PP$ such that the discounted mid-price process $\pi^{\mathrm{Up}}$ is a $\Q$-martingale:
\begin{equation}
  \pi_t^{\mathrm{Up}} \;=\; \E^{\Q}\!\left[ Y^{\mathrm{Up}}\,\middle|\,\mathcal{F}_t \right],\qquad t\in[0,t_1].
\end{equation}
Equivalently, $\pi^{\mathrm{Up}}$ is the $L^2(\Q)$-projection of the terminal payoff onto $\mathcal{F}_t$.
\end{assumption}

\begin{remark}[Incompleteness and operational interpretation]
Prediction markets are typically \emph{incomplete}; $\Q$ need not be unique.
Operationally, FAST15M treats the observed market mid as defining (one) admissible pricing measure via the identity above, and then uses structural information from the crypto price process to produce an \emph{alternative} conditional probability estimate.
The edge computation is then performed against \emph{tradable} bid/ask prices with transaction costs.
\end{remark}

\subsection{Transaction costs: bid/ask and the 15-minute fee functional}

Let $a_t^{\mathrm{Up}}$ and $a_t^{\mathrm{Down}}$ be the best asks (top-of-book) for Up/Down.
FAST15M enters positions by buying at the best ask with IOC (Immediate-or-Cancel) semantics.

\begin{assumption}[Polymarket 15-minute fee schedule]\label{ass:polymarket_fee}
For 15-minute crypto markets, Polymarket charges a deterministic fee depending on the execution price $p\in(0,1)$.
The fee per share is
\begin{equation}
  f(p) \;:=\; \frac{1}{4}\,\bigl(p(1-p)\bigr)^2\,\,\,\text{USDC/share}.
\end{equation}
Thus for a trade of $q$ shares executed at price $p$, the fee is $q\,f(p)$.
\end{assumption}

\begin{remark}
The function $f$ is symmetric about $p=1/2$, maximal at $p=1/2$ with $f(1/2)=0.015625$ (1.5625\% of notional per share), and decays toward 0 near $p\in\{0,1\}$.
The user-provided fee table is consistent with this formula (up to rounding).
\end{remark}

\begin{table}[h]
\centering
\begin{tabular}{cccc}
\toprule
Price $p$ & Fee/share $f(p)$ & Cost (100 shares) & Fee (100 shares) \\
\midrule
0.01 & 0.0000245 & 1.00 & 0.0025 \\
0.05 & 0.0005641 & 5.00 & 0.0564 \\
0.10 & 0.0020250 & 10.00 & 0.2025 \\
0.20 & 0.0064000 & 20.00 & 0.6400 \\
0.30 & 0.0110250 & 30.00 & 1.1025 \\
0.40 & 0.0144000 & 40.00 & 1.4400 \\
0.50 & 0.0156250 & 50.00 & 1.5625 \\
0.60 & 0.0144000 & 60.00 & 1.4400 \\
0.70 & 0.0110250 & 70.00 & 1.1025 \\
0.80 & 0.0064000 & 80.00 & 0.6400 \\
0.90 & 0.0020250 & 90.00 & 0.2025 \\
0.99 & 0.0000245 & 99.00 & 0.0025 \\
\bottomrule
\end{tabular}
\caption{Polymarket 15-minute fee schedule examples (computed from $f(p)=\tfrac14(p(1-p))^2$; values match the provided examples up to rounding).}
\end{table}


%==============================================================================
\section{Underlying Price Model: Semimartingale Foundations and the GBM Approximation}
%==============================================================================

\subsection{Model class and why we start there}

The relevant mathematical object is the distribution of the terminal sign event
$\{P_{t_1} > P_{t_0}\}$ (under settlement oracle) and its approximation from the observed feed.
Modern mathematical finance places asset prices in the class of (special) semimartingales because:
(i) this is the maximal class for which stochastic integration against trading strategies is well-defined, and
(ii) the Fundamental Theorem of Asset Pricing (Delbaen--Schachermayer) is formulated in this setting.

\begin{assumption}[Observed crypto price as an It\^o semimartingale]\label{ass:semimartingale}
On $(\Omega,\mathcal{F},(\mathcal{F}_t),\PP)$, the observed price process $(P_t)_{t\in[t_0,t_1]}$ is strictly positive and admits (locally) the decomposition
\begin{equation}
  d\log P_t = \mu_t\,dt + \sigma_t\,dW_t + dJ_t,
\end{equation}
where $W$ is a Brownian motion, $\sigma_t\ge 0$ is progressively measurable, and $J$ is a pure-jump semimartingale (possibly zero).
\end{assumption}

\begin{remark}[Why ``GBM'' shows up]
If the jump term $J$ is negligible over a short horizon and $\sigma_t$ is approximately constant, then $\log P$ has approximately Gaussian increments.
Formally, the continuous local-martingale part of $\log P$ is a time-changed Brownian motion (Dambis--Dubins--Schwarz), and freezing $\sigma_t$ over $[t_{\mathrm{now}},t_1]$ yields a Gaussian increment with variance $\int_{t_{\mathrm{now}}}^{t_1}\sigma_u^2\,du$.
Thus the lognormal/GBM form is the analytically tractable first-order local approximation in the semimartingale model class.
FAST15M uses it because it permits closed-form digital probabilities and microsecond-scale recomputation.
\end{remark}

\subsection{Local driftless GBM approximation (implemented baseline)}

\begin{assumption}[Local driftless GBM approximation under a pricing measure]\label{ass:gbm}
Fix a pricing measure $\Q$ as in Assumption \ref{ass:pricing_measure}.
On the short horizon $[t_{\mathrm{now}},t_1]$ the implementation approximates
\begin{equation}
    \frac{dP_t}{P_t} \approx \sigma \, dW_t^{\Q}
\end{equation}
with constant $\sigma>0$.
\end{assumption}

\begin{remark}[Why we drop drift at 15-minute horizon]
Even under a physical measure with drift $\mu$, the drift contribution to log-returns is $O(\tau)$ while the diffusion contribution is $O(\sqrt{\tau})$.
For $\tau \le 900$ seconds and typical intraday crypto volatilities, $\tfrac12\sigma^2\tau$ and $\mu\tau$ are several orders of magnitude smaller than the diffusive standard deviation $\sigma\sqrt{\tau}$.
This justifies a local driftless approximation for the specific decision statistic used (a sign event).
\end{remark}

\begin{proposition}[Log-Price Distribution]\label{prop:logprice}
Under Assumption \ref{ass:gbm}, for any $t > t_{\text{now}}$:
\begin{equation}
    \ln\left(\frac{P_t}{P_{\text{now}}}\right) \sim \mathcal{N}\left(-\frac{1}{2}\sigma^2(t - t_{\text{now}}), \sigma^2(t - t_{\text{now}})\right)
\end{equation}
\end{proposition}

\begin{proof}
By It\^o's lemma applied to $X_t = \ln P_t$:
\begin{equation}
    dX_t = -\frac{1}{2}\sigma^2 dt + \sigma \, dW_t^{\Q}
\end{equation}
Integrating from $t_{\text{now}}$ to $t$ gives the result.
\end{proof}

\subsection{Volatility Estimation}

\begin{assumption}[Realized Volatility Estimation]\label{ass:vol}
The volatility parameter $\sigma$ (per $\sqrt{\text{second}}$) is estimated from recent high-frequency price data using realized volatility:
\begin{equation}
    \hat{\sigma}^2 = \frac{1}{n}\sum_{i=1}^{n} \left(\ln\frac{P_{t_i}}{P_{t_{i-1}}}\right)^2 \cdot \frac{1}{\Delta t}
\end{equation}
where $\{P_{t_i}\}$ are sampled prices and $\Delta t$ is the sampling interval.
\end{assumption}

\begin{remark}
In implementation, we use an exponential moving average of squared returns from the Binance price feed with a half-life calibrated to approximately 30 minutes of data.
\end{remark}

%==============================================================================
\section{Probability Estimation}
%==============================================================================

\subsection{Raw Probability Calculation}

\begin{proposition}[Raw Probability of Up]\label{prop:rawprob}
Given start price $P_0$, current price $P_{\text{now}}$, volatility $\sigma$, and remaining time $\tau$ seconds, the probability that the final price exceeds the start price is:
\begin{equation}
    p_{\text{up}}^{\text{raw}} = \Phi\left(\frac{\ln(P_{\text{now}}/P_0)}{\sigma\sqrt{\tau}}\right)
\end{equation}
where $\Phi(\cdot)$ is the standard normal cumulative distribution function.
\end{proposition}

\begin{remark}[Exact driftless-GBM digital probability vs FAST15M implementation]
Under Assumption \ref{ass:gbm} with $d\log P_t = -\tfrac12\sigma^2dt+\sigma dW_t^{\Q}$, the exact conditional probability is
\begin{equation}
  \PP^{\Q}(P_{t_1}>P_0\mid\mathcal{F}_{t_{\mathrm{now}}})
  \,=\,
  \Phi\!\left(\frac{\ln(P_{\text{now}}/P_0) - \tfrac12\sigma^2\tau}{\sigma\sqrt{\tau}}\right).
\end{equation}
The function implemented in \texttt{p\_up\_driftless\_lognormal} drops the $-\tfrac12\sigma^2\tau$ term, which is negligible at 15-minute horizons for typical intraday crypto volatilities.
\end{remark}

\begin{proof}
We seek $\PP^{\Q}[P_{t_1} > P_0 \mid P_{\text{now}}]$.

By Proposition \ref{prop:logprice}:
\begin{equation}
    \ln\left(\frac{P_{t_1}}{P_{\text{now}}}\right) \sim \mathcal{N}\left(-\frac{1}{2}\sigma^2\tau, \sigma^2\tau\right)
\end{equation}

Therefore:
\begin{equation}
    \ln\left(\frac{P_{t_1}}{P_0}\right) = \ln\left(\frac{P_{\text{now}}}{P_0}\right) + \ln\left(\frac{P_{t_1}}{P_{\text{now}}}\right) \sim \mathcal{N}\left(\ln\frac{P_{\text{now}}}{P_0} - \frac{1}{2}\sigma^2\tau, \sigma^2\tau\right)
\end{equation}

The probability that $P_{t_1} > P_0$ is:
\begin{align}
    \PP^{\Q}[P_{t_1} > P_0] &= \PP^{\Q}\left[\ln\frac{P_{t_1}}{P_0} > 0\right] \\
    &= \PP^{\Q}\left[Z > \frac{-\ln(P_{\text{now}}/P_0) + \frac{1}{2}\sigma^2\tau}{\sigma\sqrt{\tau}}\right]
\end{align}
where $Z \sim \mathcal{N}(0,1)$.

For short horizons where $\frac{1}{2}\sigma^2\tau \ll |\ln(P_{\text{now}}/P_0)|$, this simplifies to:
\begin{equation}
    p_{\text{up}}^{\text{raw}} \approx \Phi\left(\frac{\ln(P_{\text{now}}/P_0)}{\sigma\sqrt{\tau}}\right)
\end{equation}
\end{proof}

\begin{corollary}[Limiting Behavior]
As $\tau \to 0$:
\begin{equation}
    p_{\text{up}}^{\text{raw}} \to \begin{cases}
        1 & \text{if } P_{\text{now}} > P_0 \\
        0.5 & \text{if } P_{\text{now}} = P_0 \\
        0 & \text{if } P_{\text{now}} < P_0
    \end{cases}
\end{equation}
At market start ($P_{\text{now}} = P_0$): $p_{\text{up}}^{\text{raw}} = 0.5$.
\end{corollary}

\subsection{RN-JD enhancement: belief volatility, martingale drift correction, and jump regime gating}\label{sec:rnjd}

The polling FAST15M implementation additionally computes an \emph{RN-JD corrected} probability estimate using the market mid as an anchor and a belief-volatility model in log-odds space.
This follows the code in \texttt{vault/rnjd.rs} and \texttt{vault/belief\_vol.rs}.

\paragraph{Market-implied probability and log-odds.}
Let $p_t^{\mathrm{mkt}}\in(0,1)$ denote a market-implied probability proxy computed from top-of-book asks:
\begin{equation}
  p_t^{\mathrm{mkt}} :=
  \begin{cases}
    \tfrac12\bigl(a_t^{\mathrm{Up}} + (1-a_t^{\mathrm{Down}})\bigr), & a_t^{\mathrm{Up}},a_t^{\mathrm{Down}}\ \text{available}\\
    a_t^{\mathrm{Up}}, & a_t^{\mathrm{Down}}\ \text{missing}\\
    1-a_t^{\mathrm{Down}}, & a_t^{\mathrm{Up}}\ \text{missing}\\
    \tfrac12, & \text{both missing.}
  \end{cases}
\end{equation}
Define the log-odds process $x_t := \logit(p_t^{\mathrm{mkt}})$.

\paragraph{Belief volatility tracking.}
Given observations $(x_{t_i})$, the implementation forms increments $\Delta x_i := x_{t_i}-x_{t_{i-1}}$ and annualizes the squared increment
\begin{equation}
  \mathrm{sqinc}_i := \frac{(\Delta x_i)^2}{\Delta t_i^{\mathrm{yr}}},
  \qquad \Delta t_i^{\mathrm{yr}} := \frac{t_i-t_{i-1}}{365.25\cdot 24\cdot 3600}.
\end{equation}
It maintains an EMA variance estimate $v_i$ and belief volatility $\sigma_b$:
\begin{equation}
  v_i := \alpha\,\mathrm{sqinc}_i + (1-\alpha)v_{i-1},\qquad \sigma_b := \sqrt{v_i},
\end{equation}
with defaults $\alpha=0.1$, prior $\sigma_b=2.0$, and a minimum of 30 samples before deeming the estimate reliable.

\paragraph{Price-vol to belief-vol conversion (bridge).}
When belief-vol estimates are unreliable, the implementation derives a belief-vol proxy from observed price volatility via
\begin{equation}
  \sigma_b^{\mathrm{px}} \approx \frac{\sigma_{\mathrm{px}}}{p(1-p)},\qquad p:=p_t^{\mathrm{mkt}}\in[0.05,0.95],
\end{equation}
and blends tracked and price-derived values (70\% tracked / 30\% price-derived when tracked is reliable).

\paragraph{Risk-neutral drift correction in log-odds.}
Under the RN-JD modeling choice, $p_t^{\mathrm{mkt}}$ is treated as a $\Q$-martingale (Assumption \ref{ass:pricing_measure}).
Writing dynamics in log-odds space pins down the drift (no-arbitrage constraint). The implemented diffusion-only drift is
\begin{equation}
  \mu_x(p) = -\tfrac12\sigma_b^2\,(1-2p),\qquad p\in(0,1),
\end{equation}
and the corresponding short-horizon drift correction is $\Delta x \approx \mu_x(p_t^{\mathrm{mkt}})\,\tau^{\mathrm{yr}}$.
Given a raw price-based probability $p_{\text{up}}^{\text{raw}}$ (Proposition \ref{prop:rawprob}), the RN-JD corrected estimate is
\begin{equation}
  p_{\text{up}}^{\mathrm{rnjd}} := \sigmoid\Bigl(\logit(p_{\text{up}}^{\text{raw}}) + \mu_x(p_t^{\mathrm{mkt}})\,\tau^{\mathrm{yr}}\Bigr).
\end{equation}

\paragraph{Jump regime detection and trading conservatism.}
The implementation computes a z-score for recent moves in $x$:
\begin{equation}
  z := \frac{|\Delta x|}{\sigma_b\sqrt{\Delta t^{\mathrm{yr}}}}.
\end{equation}
If at least 2 jumps with $z>3$ occur in the past 300 seconds, it flags \texttt{jump\_regime=true} and increases the minimum required edge by a factor of 2.
This is an explicit microstructure/jump-aware guardrail rather than a full jump-diffusion calibration.


\subsection{Conservative Shrinkage}

\begin{assumption}[Model Uncertainty]\label{ass:uncertainty}
Our probability estimate is subject to uncertainty from:
\begin{enumerate}
    \item Volatility estimation error
    \item Deviation from the GBM assumption (fat tails, jumps)
    \item Market microstructure effects not captured by the model
\end{enumerate}
\end{assumption}

To account for this uncertainty, we apply a conservative shrinkage toward the uninformative prior of 0.5:

\begin{definition}[Shrinkage Transformation]\label{def:shrink}
Let $\widehat{p}_{\text{up}}$ denote the ``base'' model probability estimate, i.e.
$\widehat{p}_{\text{up}} \in \{p_{\text{up}}^{\text{raw}},\,p_{\text{up}}^{\mathrm{rnjd}}\}$ depending on the engine variant.
The shrunk probability estimate is:
\begin{equation}
    p_{\text{up}} = 0.5 + s \cdot (\widehat{p}_{\text{up}} - 0.5)
\end{equation}
where $s \in [0, 1]$ is the shrinkage parameter.
\end{definition}

\begin{proposition}[Shrinkage Properties]
The shrinkage transformation satisfies:
\begin{enumerate}
    \item \textbf{Bounds preservation}: $p_{\text{up}} \in (0, 1)$ for all $p_{\text{up}}^{\text{raw}} \in (0, 1)$
    \item \textbf{Symmetry}: $p_{\text{up}}(0.5 + \delta) + p_{\text{up}}(0.5 - \delta) = 1$ for all $\delta$
    \item \textbf{Monotonicity}: $p_{\text{up}}$ is strictly increasing in $p_{\text{up}}^{\text{raw}}$
    \item \textbf{Conservatism}: $|p_{\text{up}} - 0.5| \leq |p_{\text{up}}^{\text{raw}} - 0.5|$
\end{enumerate}
\end{proposition}

\begin{assumption}[Default Shrinkage]\label{ass:shrink}
We use $s = 0.35$ as the default shrinkage parameter, implying that our final estimate moves only 35\% of the way from 0.5 toward the raw model estimate.
\end{assumption}

\begin{remark}
The choice of $s = 0.35$ is motivated by:
\begin{itemize}
    \item Backtesting showing improved Sharpe ratio compared to $s = 1$ (no shrinkage)
    \item Bayesian interpretation: blending a prior of $\text{Beta}(1,1)$ with observed evidence
    \item Robustness to volatility misestimation
\end{itemize}
\end{remark}

\subsection{Numerical Bounds}

\begin{assumption}[Probability Clamping]\label{ass:clamp}
To avoid numerical issues and extreme positions, probabilities are clamped:
\begin{equation}
    p_{\text{up}} \gets \max(0.0001, \min(0.9999, p_{\text{up}}))
\end{equation}
\end{assumption}

%==============================================================================
\section{Edge Calculation}
%==============================================================================

\begin{definition}[Edge]
The economically relevant comparison is between our probability estimate and the \emph{all-in} cost of acquiring a share.
With best ask $a\in(0,1)$ and fee function $f$ from Assumption \ref{ass:polymarket_fee}, the all-in buy cost per share is
\begin{equation}
  c(a) := a + f(a).
\end{equation}
Define fee-adjusted edges:
\begin{align}
    \text{edge}_{\text{up}}^{\mathrm{fee}} &= p_{\text{up}} - c(a_{\text{up}}) \\
    \text{edge}_{\text{down}}^{\mathrm{fee}} &= (1 - p_{\text{up}}) - c(a_{\text{down}})
\end{align}
\end{definition}

\begin{proposition}[Fee-adjusted edge equals expected net value of entry-only hold]\label{prop:edge_ev}
Consider buying 1 share of the Up contract at ask $a$ and holding to settlement (no exit trade).
Under Assumption \ref{ass:polymarket_fee}, the total entry cost is $c(a)=a+f(a)$ and the terminal payoff is $Y^{\mathrm{Up}}\in\{0,1\}$.
If $p:=\E^{\Q}[Y^{\mathrm{Up}}\mid\mathcal{F}_t]$ is the relevant pricing probability, then the conditional expected net value is
\begin{equation}
  \E^{\Q}[Y^{\mathrm{Up}}-c(a)\mid\mathcal{F}_t] = p-c(a)=\text{edge}_{\text{up}}^{\mathrm{fee}}.
\end{equation}
\end{proposition}

\begin{remark}[Implementation note]
The current FAST15M entry logic in \texttt{engine.rs} and \texttt{fast15m\_reactive.rs} compares $p$ against the raw ask (i.e. uses $p-a$) and enforces a minimum edge threshold.
Economically, the threshold should be interpreted as covering $f(a)$ plus spread/slippage and model risk.
\end{remark}

%==============================================================================
\section{Microstructure and Execution Model (Polymarket CLOB + Oracle Basis Risk)}
%==============================================================================

\subsection{Orderbook primitives and the tradable price}

Polymarket's CLOB is an order-driven market with a discrete limit order book.
At any time $t$, for each outcome token, the book induces a best bid $b_t$ and best ask $a_t$.
FAST15M uses the best ask $a_t$ as the executable entry price (buying liquidity).

\begin{assumption}[Top-of-book execution model]\label{ass:tob_exec}
On entry, the strategy submits an IOC buy order at the current best ask and assumes (up to partial fill) execution at a price close to $a_t$.
Any residual non-filled quantity is cancelled.
\end{assumption}

\subsection{Market data access, staleness, and the ``HFT cache'' contract}

The implementation maintains a WebSocket-driven orderbook cache for Polymarket tokens.
Let $\widehat{\mathcal{B}}_t$ denote the cached book snapshot and $\Delta_{\mathrm{stale}}$ the maximum staleness threshold.

\begin{assumption}[Staleness-bounded best-ask reads]\label{ass:stale}
Best-ask reads prefer the WS cache; if the cached snapshot is newer than $\Delta_{\mathrm{stale}}\approx 1500\,\mathrm{ms}$, the best ask is extracted from it.
If the cache is stale or missing and the path is not designated HFT-only, a REST snapshot fallback may be used with a short timeout.
\end{assumption}

\begin{remark}
This is a microstructure-critical assumption: when the cache is stale, using an ask from a REST snapshot introduces additional latency and selection bias.
The codebase also contains a cache-only best-ask helper (``HFT-grade''), which never blocks on REST and instead skips the tick.
\end{remark}

\subsection{Execution costs in simulation vs reality}

\begin{assumption}[Paper execution adapter cost model]\label{ass:paper_costs}
In paper mode, execution cost is modeled via:
\begin{itemize}
  \item random latency (base + jitter),
  \item adverse slippage (base spread-crossing + size-dependent impact),
  \item partial fills with a fixed probability, and
  \item a constant fee rate (default 0.5\% of notional) applied to filled notional.
\end{itemize}
\end{assumption}

\begin{remark}
Assumption \ref{ass:paper_costs} is a \emph{simulation} model. Real Polymarket 15-minute fees follow Assumption \ref{ass:polymarket_fee} (price-dependent quadratic fees).
From a specification standpoint, the economically correct fee functional is $f(p)$; the constant 0.5\% is an approximation used by the paper adapter.
\end{remark}

\subsection{Oracle basis risk (Chainlink vs Binance) and a guardrail}

\begin{assumption}[Oracle divergence guardrail (polling FAST15M path)]\label{ass:oracle_guard}
Let $P_t$ be Binance mid and $P_t^{\mathrm{CL}}$ be Chainlink.
When the Chainlink feed is available, the polling FAST15M path may skip trading if estimated divergence (in bps) and/or oracle staleness enters a dangerous regime.
\end{assumption}


\begin{assumption}[Minimum Edge Threshold]\label{ass:minedge}
We only consider trading when edge exceeds a minimum threshold $e_{\min}$:
\begin{equation}
    \text{edge} > e_{\min}
\end{equation}
Default: $e_{\min} = 0.01$ (1\%).
\end{assumption}

\begin{remark}
The minimum edge requirement accounts for:
\begin{itemize}
    \item Polymarket 15-minute fees $f(a)$ (Assumption \ref{ass:polymarket_fee}), which peak at $\approx 1.56\%$ per trade near $a=0.5$
    \item Slippage from crossing the spread
    \item Model estimation error
\end{itemize}
\end{remark}

%==============================================================================
\section{Position Sizing: Kelly Criterion}
%==============================================================================

\subsection{Kelly Formula for Binary Outcomes}

\begin{definition}[Kelly Fraction]
For a binary bet where we estimate probability $p$ of winning and the market offers odds implied by price $q$ (i.e., decimal odds $b = 1/q - 1$), the optimal fraction of bankroll to wager is:
\begin{equation}
    f^* = \frac{p \cdot b - (1-p)}{b} = \frac{p(1/q - 1) - (1-p)}{1/q - 1}
\end{equation}
\end{definition}

\begin{proposition}[Kelly Simplification for Binary Markets]
When buying at price $q$ with estimated probability $p$:
\begin{equation}
    f^* = \frac{p - q}{1 - q}
\end{equation}
\end{proposition}

\begin{proof}
The decimal odds when buying at price $q$ are $b = (1-q)/q$ (you pay $q$ to win $1-q$ net). Substituting:
\begin{align}
    f^* &= \frac{p \cdot \frac{1-q}{q} - (1-p)}{\frac{1-q}{q}} \\
    &= \frac{p(1-q) - q(1-p)}{1-q} \\
    &= \frac{p - pq - q + pq}{1-q} \\
    &= \frac{p - q}{1 - q}
\end{align}
\end{proof}

\subsection{Fractional Kelly}

\begin{assumption}[Fractional Kelly]\label{ass:fractkelly}
To reduce variance and account for estimation error in $p$, we use fractional Kelly:
\begin{equation}
    f_{\text{actual}} = \kappa \cdot f^*
\end{equation}
where $\kappa \in (0, 1]$ is the Kelly fraction multiplier.
\end{assumption}

\begin{remark}
The choice of $\kappa$ trades off expected growth rate against variance:
\begin{itemize}
    \item $\kappa = 1$: Full Kelly, maximum expected log-growth but high variance
    \item $\kappa = 0.5$: Half Kelly, 75\% of expected growth with 50\% of variance
    \item $\kappa = 0.25$: Quarter Kelly, more conservative
\end{itemize}
We use $\kappa = 0.05$ for FAST15M due to the high frequency of trades.
\end{remark}

\subsection{Position Limits}

\begin{assumption}[Maximum Position]\label{ass:maxpos}
Individual positions are capped at a maximum fraction of bankroll:
\begin{equation}
    f_{\text{actual}} \gets \min(f_{\text{actual}}, f_{\max})
\end{equation}
Default: $f_{\max} = 0.01$ (1\% of bankroll per trade).
\end{assumption}

\begin{assumption}[Minimum Position]\label{ass:minpos}
Positions below a minimum USD threshold are not executed:
\begin{equation}
    \text{Position}_{\text{USD}} = B \cdot f_{\text{actual}} \geq \text{pos}_{\min}
\end{equation}
Default: $\text{pos}_{\min} = \$1$.
\end{assumption}

%==============================================================================
\section{Implemented FAST15M Decision Logic (Polling and Reactive Variants)}
%==============================================================================

This repository contains two concrete implementations of the 15-minute Up/Down entry logic.
Both are ``entry-only'' (buy at ask, hold to resolution) and differ primarily in (i) evaluation cadence and latency targets, and (ii) probability model enrichment.

\subsection{Variant A: Polling FAST15M (\texttt{VaultEngine::evaluate\_updown15m})}

\begin{algorithm}[H]
\caption{Polling FAST15M decision (RN-JD enhanced + vol-adjusted Kelly)}
\label{alg:fast15m_polling}
\begin{algorithmic}[1]
\Require Parsed 15m market window $(t_0,t_1)$, token IDs (Up/Down)
\Require Binance mid $P_{\text{now}}$, start proxy $P_0$, realized vol $\sigma$ (per $\sqrt{\text{s}}$)
\Require Best asks $a_{\text{up}}, a_{\text{down}}$ from WS cache (REST fallback allowed)
\Require Parameters: shrink $s$, min edge $e_{\min}$, cooldown, bankroll/risk caps
\Ensure IOC buy order request or NO\_TRADE

\State $\tau \gets (t_1- t_{\text{now}})$
\If{$\tau < 15\text{s}$} \Return NO\_TRADE \EndIf
\State \textbf{(Oracle guard)} If Chainlink divergence/staleness is dangerous, \Return NO\_TRADE

\State Compute market mid proxy $p^{\mathrm{mkt}}$ from $(a_{\text{up}},a_{\text{down}})$ and record it in the belief-vol tracker

\State \textbf{(RN-JD estimate)} $\widehat{p}_{\text{up}} \gets p_{\text{up}}^{\mathrm{rnjd}}$ using Section~\ref{sec:rnjd} (fallback: $p_{\text{up}}^{\text{raw}}$)
\State \textbf{(Shrink)} $p_{\text{up}} \gets 0.5 + s(\widehat{p}_{\text{up}}-0.5)$; $p_{\text{down}}\gets 1-p_{\text{up}}$

\State \textbf{(Side selection)} choose Up vs Down by comparing $p_{\text{up}}-a_{\text{up}}$ and $p_{\text{down}}-a_{\text{down}}$
\State $\text{edge} \gets p_{\text{side}} - a_{\text{side}}$
\State \textbf{(Jump regime gating)} If jump regime, set $e_{\min}\gets 2e_{\min}$
\If{$\text{edge} < e_{\min}$} \Return NO\_TRADE \EndIf

\State \textbf{(Vol-adjusted Kelly)} compute $\sigma_b$ from belief-vol tracker and size via \texttt{kelly\_with\_belief\_vol}
\If{Kelly says skip or position $<\$1$} \Return NO\_TRADE \EndIf

\State \Return IOC BUY at price $a_{\text{side}}$ for notional determined by Kelly
\end{algorithmic}
\end{algorithm}

\subsection{Variant B: Reactive FAST15M (\texttt{ReactiveFast15mEngine})}

\begin{algorithm}[H]
\caption{Reactive FAST15M decision (event-driven + cache-only book)}
\label{alg:fast15m_reactive}
\begin{algorithmic}[1]
\Require A Binance price update event for one of \{BTC,ETH,SOL,XRP\}
\Ensure Optional IOC buy order request or NO\_TRADE

\State Compute current 15m window start $t_0 := t_{\text{now}} - (t_{\text{now}}\bmod 900)$ and end $t_1 := t_0+900$
\If{$t_1-t_{\text{now}} < 60\text{s}$} \Return NO\_TRADE \EndIf
\State Enforce per-asset cooldown (default 30s)
\State Enforce a minimum evaluation interval outside the first 60s of the window

\State Estimate $p_{\text{up}}^{\text{raw}}$ via \texttt{p\_up\_driftless\_lognormal} and apply shrinkage $s$
\State Apply ``edge-change'' gating: only proceed if $|\,|p_{\text{up}}-0.5|-\text{last\_edge}\,|$ exceeds threshold

\State Resolve Up/Down token IDs via Gamma and cache them
\State Fetch best asks from the WS book cache \emph{only} (skip tick if cache miss/stale)

\State Choose side by comparing $p-a$ for Up/Down
\If{$\text{edge}<e_{\min}$} \Return NO\_TRADE \EndIf

\State Size via standard fractional Kelly \texttt{calculate\_kelly\_position}
\If{Kelly says skip} \Return NO\_TRADE \EndIf

\State \Return IOC BUY at the best ask for the selected side
\end{algorithmic}
\end{algorithm}

%==============================================================================
\section{Operational Constraints}
%==============================================================================

\subsection{Cooldown Period}

\begin{assumption}[Trade Cooldown]\label{ass:cooldown}
After executing a trade on an asset, no further trades on that asset are permitted for $T_{\text{cool}}$ seconds.
\begin{equation}
    t_{\text{current}} - t_{\text{last\_trade}} \geq T_{\text{cool}}
\end{equation}
Default: $T_{\text{cool}} = 30$ seconds.
\end{assumption}

\begin{remark}
The cooldown prevents overtrading when price oscillates around a threshold and ensures sufficient time for market conditions to evolve.
\end{remark}

\subsection{Window Boundaries}

\begin{assumption}[Window-end skip buffers]\label{ass:window}
To avoid adverse selection and settlement/latency risk near expiry, the implementations enforce a window-end skip buffer.
Let $\tau=t_1-t_{\mathrm{now}}$.
\begin{itemize}
  \item \textbf{Polling FAST15M:} skip if $\tau < 15\,\mathrm{s}$.
  \item \textbf{Reactive FAST15M:} skip if $\tau < 60\,\mathrm{s}$ (\texttt{window\_end\_idle\_sec}).
\end{itemize}
\end{assumption}

\begin{remark}
This constraint avoids:
\begin{itemize}
    \item Extreme probability estimates near expiry (as $\tau \to 0$)
    \item Execution risk when settlement is imminent
    \item Reduced liquidity typically observed near expiry
\end{itemize}
\end{remark}

%==============================================================================
\section{Summary of Assumptions}
%==============================================================================

\subsection{Mathematical finance / prediction-market setting}

\begin{enumerate}[label=\textbf{S\arabic*.}]
    \item \textbf{(Stochastic basis)} We work on $(\Omega,\mathcal{F},(\mathcal{F}_t),\PP)$ (Definition \ref{def:stochastic_basis}).
    \item \textbf{(Payoff model)} Up/Down outcome shares are bounded claims with payoffs as in Definition \ref{def:resolution_payoffs}.
    \item \textbf{(Pricing measure, idealized)} Ignoring frictions, there exists $\Q\sim\PP$ such that mid prices satisfy $\pi_t^{\mathrm{Up}}=\E^{\Q}[Y^{\mathrm{Up}}\mid\mathcal{F}_t]$ (Assumption \ref{ass:pricing_measure}).
\end{enumerate}

\subsection{Observed crypto price model and justification}

\begin{enumerate}[label=\textbf{M\arabic*.}]
    \item \textbf{(Semimartingale baseline)} $\log P$ is modeled as an It\^o semimartingale with possible jumps (Assumption \ref{ass:semimartingale}).
    \item \textbf{(Local GBM approximation)} For ultra-short horizon probability computation, we freeze volatility and drop drift, yielding a driftless GBM/lognormal approximation under a pricing measure (Assumption \ref{ass:gbm}).
    \item \textbf{(Volatility estimation)} $\sigma$ is estimated from recent high-frequency returns (Assumption \ref{ass:vol}).
    \item \textbf{(Oracle basis risk)} Settlement uses Chainlink $P^{\mathrm{CL}}$ while the model observes Binance $P$; a guardrail may skip trading under dangerous divergence (Assumption \ref{ass:oracle_guard}).
\end{enumerate}

\subsection{Microstructure and transaction costs}

\begin{enumerate}[label=\textbf{E\arabic*.}]
    \item \textbf{(Top-of-book IOC)} Entry is modeled as buying at best ask with IOC semantics (Assumption \ref{ass:tob_exec}).
    \item \textbf{(Staleness)} Best-ask reads prefer WS cache with a staleness bound; some paths allow REST fallback (Assumption \ref{ass:stale}).
    \item \textbf{(15m fee functional)} Real 15m crypto markets have price-dependent fees $f(p)=\tfrac14(p(1-p))^2$ (Assumption \ref{ass:polymarket_fee}).
    \item \textbf{(Paper costs)} Paper execution may instead approximate fees as a constant rate plus slippage/latency noise (Assumption \ref{ass:paper_costs}).
\end{enumerate}

\subsection{Strategy controls (implemented guardrails)}

\begin{enumerate}[label=\textbf{T\arabic*.}]
    \item \textbf{(Shrinkage)} Apply shrinkage toward $0.5$ with default $s=0.35$ (Definition \ref{def:shrink}).
    \item \textbf{(Minimum edge)} Require a minimum edge threshold (default 1\%); polling FAST15M doubles this in jump regime (Assumption \ref{ass:minedge} + Section~\ref{sec:rnjd}).
    \item \textbf{(Jump regime)} Detect jump regimes from log-odds z-scores and require stricter entry conditions (Section~\ref{sec:rnjd}).
    \item \textbf{(Position sizing)} Use fractional Kelly; polling FAST15M uses belief-vol-adjusted Kelly while reactive FAST15M uses standard Kelly (Assumption \ref{ass:fractkelly}).
    \item \textbf{(Position caps)} Enforce a max-position fraction and a minimum notional (Assumptions \ref{ass:maxpos}--\ref{ass:minpos}).
    \item \textbf{(Cooldown)} Enforce a per-asset cooldown (default 30s) (Assumption \ref{ass:cooldown}).
    \item \textbf{(Window-end skip)} Skip the last part of the window (15s polling; 60s reactive) (Assumption \ref{ass:window}).
\end{enumerate}

%==============================================================================
\section{Parameter Reference}
%==============================================================================

\begin{table}[h]
\centering
\begin{tabular}{llll}
\toprule
\textbf{Parameter} & \textbf{Symbol} & \textbf{Default} & \textbf{Environment Variable} \\
\midrule
Shrinkage & $s$ & 0.35 & \texttt{UPDOWN15M\_SHRINK} \\
Minimum edge & $e_{\min}$ & 0.01 & \texttt{UPDOWN15M\_MIN\_EDGE} \\
Kelly fraction & $\kappa$ & 0.05 & \texttt{UPDOWN15M\_KELLY\_FRACTION} \\
Max position \% & $f_{\max}$ & 0.01 & \texttt{UPDOWN15M\_MAX\_POSITION\_PCT} \\
Min position USD & $\text{pos}_{\min}$ & \$1 & --- \\
Cooldown & $T_{\text{cool}}$ & 30s & \texttt{UPDOWN15M\_COOLDOWN\_SEC} \\
Window end buffer & $T_{\text{end}}$ & 60s & --- \\
Poll interval & --- & 2000ms & \texttt{UPDOWN15M\_POLL\_MS} \\
\bottomrule
\end{tabular}
\caption{FAST15M Configuration Parameters}
\end{table}

%==============================================================================
\section{Conclusion}
%==============================================================================

This specification describes the 15-minute Up/Down trading logic as an application of (i) prediction-market pricing as conditional expectation on a filtered probability space, (ii) a semimartingale-to-local-diffusion modeling step for the observed crypto price process, and (iii) explicit microstructure and transaction-cost constraints from the Polymarket CLOB.

Concretely, the system implements two entry-only variants:
\begin{itemize}
  \item a low-latency reactive engine (event-driven) using a driftless lognormal approximation plus shrinkage and standard fractional Kelly, and
  \item a polling engine that enriches the estimate with an RN-JD drift correction in log-odds space, belief-volatility tracking, and jump-regime gating, and sizes with belief-vol-adjusted Kelly.
\end{itemize}

The document also makes explicit the non-negotiable ``plumbing'' assumptions: price-dependent 15-minute fees $f(p)$, staleness-bounded orderbook reads, IOC execution semantics, and oracle basis risk (Chainlink vs Binance).

\end{document}
